\section{Introduction}

Reservoir computing (RC) is a form of machine learning that sprung out from the study of recurrent neural networks (RNNs).
In short, it utilizes the dynamics of some complex system dubbed a 'reservoir' to preprocess a timeseries problem,
transforming it from a temporal to a spacial one in the reservoir, making it then separable with a usually simple readout layer.

In this paper we investigate the dynamics of Reservoir Computing systems where the reservoir is a Random Boolean Network (RBN) \cite{gershenson2004introduction},
an approach found fruitful in \cite{rbn-reservoir}.

First we create a working RBN-reservoir computing system (RRC) in the Python programming language,
reproducing chosen experiments from \cite{rbn-reservoir}.
These include finding the optimal input connectivity ($L$) and internal connectivity $K$,
as well as testing their measure of Computational Capability against actual reservoir performance.

Next we investigate whether the readout layer of a working RRC system can be re-used with other RBN-reservoirs than the one it was originally trained on,
and still stay accurate on the original classification task.
These functionally equivalent reservoirs, if any, will be evolved through the use of a genetic algorithm (GA).

Finally we look at the dynamics and characteristics of these groups of RBN-Reservoirs,
attempting to find any similarities that made them exploitable for computation.

\todo[inline]{Mention something about the many-to-one mapping of readout->reservoir leading to the possibility of using a generative genome here, or take that later?}

\subsection{A brief introduction to reservoir computing}

Recurrent neural networks, as opposed to feed-forward neural networks,
are notoriously time consuming and difficult to train.
This due to feedback from the recurrent connections during the training process,
allowing small topology changes to drastically change ones position in the fitness landscape.

It was therefore proposed both in \cite{jaeger2002adaptive} (as echo state networks, or ESN)
and \cite{natschlager2002liquid} (as liquid state machines, or LSM) to separate the RNN into two parts,
the untrained reccurrent reservoir, and the trained readout layer.
Both of these methods have been unified into the field of Reservoir Computing,
now focusing on the separate training and evolution of the recurrent and readout part \cite{lukovsevivcius2012reservoir}.

Useful RC resources include Organic \cite{organic}, an online RC hub,
providing documentation, references, and a RC toolbox implemented in python.
It has been quite useful for this project.
Exiting applications of reservoir computing include speech and handwriting recognition,
as well as controlling robotics, as detailed in \cite{lukovsevivcius2012reservoir}.

\subsection{Alternatives to classical reservoirs}

An interesting question arises:
Are there other types of complex systems that can be used as reservoirs?
What properties must these reservoirs have to be able to solve problems?

Complex networks similar to the sparsely connected RNNs found in classical RC devices include Cellular Automata and Random Boolean Networks.
Both are considerably less complex than RNNs, requiring only simple lookup tables for state transitions as opposed to multiplication for RNNs.
A short treatise on Cellular Computing \& friends is available here \cite{sipper1999emergence}.
Cellular computing provides a potentially powerful alternative to classical computers,
leveraging extreme parallelism, simple components and local state.

The ever-so-cited 'Water bucket' paper investigated the use of an actual bucket of water as a reservoir \cite{fernando2003pattern},
successfully recognizing patterns and achieving decent performance at that.
The RBN-reservoir \cite{rbn-reservoir} approach mentioned earlier, has also been found to be viable.

\subsection{Dynamics and complexity of reservoirs}

In general, one wishes to use a reservoir with a complexity matching that of the problem.
Utilizing a modern processor to predict a very simple timeseries would be a classic case of shooting sparrows with cannons \cite{wiki:sparrow},
while achieving the same using a RBN or CA might be more impressive.

In general the computational power of a complex system is the highest when the system is on the edge of chaos \cite{langton3computation}.
In \cite{rbn-reservoir}, the authors find that this also holds for RBN based reservoir systems.
They also propose a metric for predicting the computational power of RBNs,
which is found to correlate with actual performance.

In addition one can look at the attractor lengths and transient times of a RBN,
and attempt to uncover any insights that might be available there,
even though they might not correlate with actual performance due to the continious perturbance of the network.

These metrics can be used to categorize the computational power of other reservoirs,
even physical materials \cite{miller2002evolution},
and will be useful through this paper.
