\section{Introduction}

\subsection{Why this?}

\subsection{Reservoir computing}

\subsubsection{Liquid state machines and echo state networks}

\subsection{Random boolean networks}

\section{Stuff about complexity?}

\subsection{How complex must things be, edge of chaos?}

\subsection{Complexity of RBNs, how to measure}

\subsection{Complexity about Reservoir computing thingies?}

\section{Results or something?}

\subsection{Show working case of reservoir computing with normal Echo State or something}

\subsection{Put in Random boolean network?}

\subsection{Show how 'complex' the boolean network is? Show that it also can do computations.}

\section{Correspondence between complexity of RBN and computations achieved?}

\section{Discussion}

\section{Complexity measures for rbn}
Wow such success, you revolutionized the field, Aleksander!

Well fuck\cite{reservoir-rbn}.

Hvilken kodebase? Matlab rbn toolbox! \url{http://www.teuscher.ch/rbntoolbox/UsingToolbox/main\_toolbox.htm}

In RC, we continuously
perturb the reservoir and so the underlying RBN of our
model  is  not  a  closed  system.   Therefore,  computation
cannot be dependent on attractors and must be enabled
by the dynamics of the RBN. However, in some circum-
stances the network dynamics can fall into an attractor
temporarily or indefinitely, due to frozen dynamics, inad-
equate distribution of the input signal, or a non-random input  stream.   Therefore,  RC  is  a  novel  framework  in
which to explore the capacity of RBN dynamics for infor-
mation processing.

\section{Energy efficiency}
Mere ting fra 2014 \cite{reservoir-rbn-energy-efficiency}.
\begin{itemize}
  \item Interessante seksjoner: 1.2.4 Measures for RC-RBNs
\end{itemize}


To reiterate, there
has not been much research in the study of RC devices using reservoirs with
heterogenous
in-degrees,  as  most  work  is  concerned  with  the  training  of  RNNs  (normally  homogenous)
reservoirs.   The  study  of  heterogenous  reservoirs  may  give  insight  into  methods  of  using
generally undesirable physical phenomena in a computationally useful way.


The major nding of the SGT study showed that the computational capability of RBNs
was maximized for networks with critical dynamics.  The results were not obtained analyti-
cally, but reinforced previous analytic claims through the use of simulation.  The simulator
used to generate their results was likely a tool called
RBN Toolbox
created in part by Christof
Teuscher [8]


Although the issue is not
addressed in our work, research should be conducted in the development of new measures
which do not support these 
aws.  In our research though,  the chosen measures,  specically
computational capability, work well in representing the dynamics of many generated networks
as the averaged measure becomes more accurate as the sample count increases.

The choice to set L equal to N comes from previous
research in [4], where computational capability is shown to be higher for values of
L closer to N (the network is being perturbed more).
(Red.amn. L er antall koblinger fra input til RBN)

\section{Training}
Woop de har faen meg trent de også!!!
\cite{how-to-train-your-rbn}

L = N * 0.20

We conclude that RBNs can be successfully setup in solving goal-driven prob-
lems  by  interactively  training  its  read-out  layer  through  reinforcement  learning
approach.  They are particularly echocient in solving POMDPs due to the dynamics
achieved by the recurrent connections in such networks.  Though the RBN-based
agents  performed  comparable  to  other  benchmark  techniques  such  as  evolution-
ary approaches talked about in related literatures, we see that they were able to
successfully adapt to their set environments in an interactive manner.
In future work, we would like to extend our framework to memristor based hard-
ware implementations of reservoirs.  We aspire to setup memristor based reservoirs
augmented with read-out layers in reinforcement learning contexts and train them
interactively to function as autonomous agents.
