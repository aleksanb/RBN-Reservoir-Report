\section{Future work}

\subsection{Required Complexity of Tasks}

Only RBN Reservoirs of size $N=100$ are looked at in this paper,
and in \cite{rbn-reservoir} reservoirs of size $N=500$ are used.
Neither might be the optimal size for an RBN,
with the best dynamics and problem accuracies potentially obtained at different reservoir sizes.
If it turns out that a given task can be solved just as easily for a reservoir of size $N=50$ as $N=500$ one might as well use the smaller reservoir,
saving bits, hardware, and the environment in one go.
If one can obtain the relationship betwen how complex a reservoir has to be to solve a given task,
one could predict i.e. how large the water bucket reservoir presented in \cite{fernando2003pattern} actually has to be.

\subsection{Generative genomes}

A fixed-representation genome was used in this paper to evolve functionally equivalent reservoirs,
with the fitness function requiring the reservoirs to adapt to a given readout layer.
Alternatively one could evolve RBNs towards the dynamics known to be useful for computation.
This can be done with a generative genome, where the genome defines how the graph grows to its adult form,
as opposed to describing the edges and nodes of the adult RBN directly.

\subsection{Scaling up Simulation}

Software simulations of RBNs and the training of the corresponding RRC systems can be rather slow.
Therefore it might be more efficient to implement the RRC system in an FPGA,
or as an accellerated task on a supercomputer or graphics card.

\section{Conclusion}

A functioning RBN Reservoir Computing system was implemented,
and its results validated against and found in accordance with those from a previous publication:
A positive correlation between the computational capability of a reservoir and its actual performance is found.
The optimal connectivity for homogenous reservoirs is found to be $K=3$ as opposed to $\langle K \rangle = 2$ for heterogenous reservoirs.
Finally, the required input connectivity is found to rise with the presence of chaotic dynamics in the reservoir.

A genetic algorithm is created for evolving functionally equivalent reservoirs for use under the same readout layer in RRC systems.
There turns out to be a great number of functionally equivalent reservoirs, and finding them is easy and efficient.
A many-to-one mapping between reservoirs and allready trained readout layers is therefore present.
Their dynamics and properties are however more representative of the general reservoir population than the original reservoir.
This makes the potential use of a smaller generative genome for evolving RRC systems interesting.
Even though it hits fewer points in the RBN fitness landscape than the fixed genome used in this paper,
a large amount of these points are still usable for each instance of a working readout layer.

The use of the relatively simple Random Boolean Network in Reservoir Computing is a relatively new approach,
but one found fruitful.
The implemented RRC and GA systems are generic and reusable and verified to work according to specification,
opening up for further research within the field during the authors master's thesis.
