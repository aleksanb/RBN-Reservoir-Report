\section{Future work}

\subsection{How complex do the reservoirs actually have to be?}

Can they be really small, and still adequately do the task?
The time was not there to research this for this preliminary project.

\subsection{Using generative genomes}

Fixed-representation genomes of graphs are hella space-inefficient.
A generative genome would instead represent the growing of the graph,
but would be able to 'hit' fewer targets in adult state-space.
Therefore, the finding that there are multiple acceptable RBN-Reservoirs per readout layer
comforts us that a generative genome could also be efficient.

\subsection{Scaling up simulation}

RBN-simulation in python is hella slow.
FPGA / supercomputers instead?

\section{Conclusion}

Turns out you can evolve RBN-Reservoirs to use similar readout layers!
Yay, this is cool as that indicates that there might be exploitable properties of material that can be used to figure out if something is good for computation, before having to use it for computations!

More stuff here later i guess
