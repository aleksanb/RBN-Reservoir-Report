\section{Future work}

\subsection{Required Complexity of Tasks}

Only RBN Reservoirs of size $N=100$ are looked at in this paper,
and in \cite{rbn-reservoir} reservoirs of size $N=500$ are used.
Neither might be the optimal size for an RBN,
with the best dynamics and problem accuracies potentially obtained at different reservoir sizes.
If it turns out that a given task can be solved just as easily for a reservoir of size $N=50$ as $N=500$ one might as well use the smaller reservoir,
saving bits, hardware, and the environment in one go.
If one can obtain the relationship betwen how complex a reservoir has to be to solve a given task,
one could predict i.e. how large the water bucket reservoir presented in \cite{fernando2003pattern} actually has to be.

\subsection{Generative genomes}

A fixed-representation genome was used in this paper to evolve functionally equivalent reservoirs,
with the fitness function requiring the reservoirs to adapt to a given readout layer.
Alternatively one could evolve RBNs towards the dynamics known to be useful for computation.
This can be done with a generative genome, where the genome defines how the graph grows to its adult form,
as opposed to describing the edges and nodes of the adult RBN directly.

\subsection{Scaling up Simulation}

Software simulations of RBNs and the training of the corresponding RRC systems can be rather slow.
Therefore it might be more efficient to implement the RRC system in an FPGA,
or as an accellerated task on a supercomputer or graphics card.

\section{Conclusion}



\todo[inline]{To be finished}

Turns out you can evolve RBN-Reservoirs to use similar readout layers!
Yay, this is cool as that indicates that there might be exploitable properties of material that can be used to figure out if something is good for computation, before having to use it for computations!

More stuff here later i guess

\todo[inline]{GOTTA FIX DIS FAST}
\subsection{Dynamics and complexity of reservoirs}

In general, one wishes to use a reservoir with a complexity matching that of the problem.
Utilizing a modern processor to predict a very simple timeseries would be a classic case of shooting sparrows with cannons \cite{wiki:sparrow},
while achieving the same using a RBN or CA might be more impressive.

The computational power of a complex system is the highest when the system is on the edge of chaos \cite{langton3computation}.
In \cite{rbn-reservoir}, the authors find that this also holds for RBN based reservoir systems.
They also propose a metric for predicting the computational power of RBNs,
which is found to correlate with actual performance,
and will be used throughout this paper.
One could also look at the attractor lengths and transient times of a RBN,
attempting to uncover any insights that might be available there.
The continious perturbance of the reservoir however lessens the importance of attractors,
as the chance of settling into one is decreased.
It will therefore not be used.
