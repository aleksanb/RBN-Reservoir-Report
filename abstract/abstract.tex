\begin{abstract}

I will rewrite this abstract to better fit what i've actually done this project.
Expect updates in this section around 13:00 22. december.

%Complexity and dynamics of RBN as benchmark for  Reservoir computing
%Reservoir computing is a relatively new approach to exploit echo state networks and liquid state-machines for computation. However, any sparsely connected network that include feedforward and feedback loops may be exploited as a reservoir. Random-Boolean-Networks (RBN) is such a sparsely connected network that may be suitable for reservoir computing. RBN is closely connected to complex systems and have been studied with in this framework. As such, classification and known properties of RBN may be used to investigate complexity and dynamics of the reservoir toward ability to solve computational problems.

%In this project the goal is to investigate properties of RBNs, e.g. complexity, dynamics, and link these to a reservoir computing systems ability to solve computational problems. Further, such information can be used to propose reservoirs based on physical materials, e.g. liquid crystals or carbon nanotubes.

\end{abstract}
