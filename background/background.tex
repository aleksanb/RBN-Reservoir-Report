\section{Background}

\subsection{A brief introduction to boolean networks}

Random Boolean Networks, also known as Kaffman networks, were originally developed as a model of genetic regulatory networks \cite{kauffman1969metabolic}.
The model makes no assumptions about the actual functioning of the nodes in the network,
making them useful to model a wide range of phenomena.
The simplification of a system to a boolean model doesn't pose a problem,
as any multi-valued network can be transformed to a corresponding binary one.

A RBN is usually described by its number of nodes $N$ and the in-degree $K$ of the nodes,
that is, how many nodes each node depends on (also known as a nodes ancestors).
RBNs can have both homogenous and heterogenous in-degrees.
In heterogenous networks, one usually describes the average connectivity $<K>$ instead.

Each node can have a state of zero or one.
The next state of the node is solely determined by the current combination of states of its ancestors.
Each combination leads to a new state of zero or one,
with the probability given by a binomial distribution usually having $<p>=0.5$.

In the classical RBN model (CRBN), all nodes update at the same time,
therefore the states of the network at $t+1$ only depends on the states at $t$.
This is a simplification that isn't quite accurate for all systems,
notably gene regulation networks where the system doesn't operate in lockstep.
There are therefore a number of updating schemes on the spectrum of determinism and randomness.

Talk about phases, ordered, chaotic. K=2 critical.

K, p, combinations giving critical dynamics. ONly statistical averages,
not necessarily representative for actual networks.

RBNs 

\begin{figure}
  \subfloat[Transition table for node a]{
    \begin{tabular}{ c c | c}
      \multicolumn{2}{c}{Ancestor states} & New state \\
      \hline
      0 & 0 & 1 \\
      0 & 1 & 1 \\
      1 & 0 & 0 \\
      1 & 1 & 1 \\
      0 & 0 & 0 \\
      0 & 1 & 1 \\
      1 & 0 & 0 \\
      1 & 1 & 0 \\
    \end{tabular}
  }
  \subfloat[Topology]{
    \begin{tikzpicture}[node distance=2cm]
      \node[vertex] (a) {a};
      \node[vertex] (b) [below left of=a] {b};
      \node[vertex] (c) [below right of=a] {c};

      \draw[edge] (a) to[bend right] (b);
      \draw[edge] (a) to (c);

      \draw[edge] (b) to (a);
      \draw[edge] (b) to[bend right] (c);

      \draw[edge] (c) to[bend right] (a);
      \draw[edge] (c) to (b);
    \end{tikzpicture}
  }
  \caption{A simple RBN with N=3, K=2.}
\end{figure}


We use \cite{gershenson2004introduction} here and prob the original kaufmann papers.
Talk about lambda n shit i guess.


\subsection{Liquid state machines / Echo state networks}

\subsection{Using boolean networks in reservoir computing}

Can we use these simple networks for computation?
Turns out Yes!, as shown in this paper \cite{rbn-reservoir}, and it works reasonably well.

Takeaways from the paper:
\begin{itemize}
  \item Relationship between dynamical properties and computational power
  \item Required connections to input layer, how much perturbation required?
\end{itemize}

Advantages include being MUCH less complex than Echo-state networks and similar, as those require operations such as 'multiplication' which is orders more complex than the simple lookup-table transitions for the RBN nodes.


\begin{itemize}
  \item LSM / ESN
  \item 2013-RBN Paper
  \item RBN-Kauffmann networks
  \item Sipper programming complexity (Fra TDT1/22) Snakk om vanskeligheten av å programmere
\end{itemize}
