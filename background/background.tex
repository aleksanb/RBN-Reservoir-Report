\section{Background}

\subsection{Liquid state machines / Echo state networks}

\subsection{An introduction to random boolean networks}

Kaufmann networks, N, K
\todo[inline]{Show kaufmann network}

\subsection{Using boolean networks in reservoir computing}

Can we use these simple networks for computation?
Turns out Yes!, as shown in this paper \cite{rbn-reservoir}, and it works reasonably well.

Takeaways from the paper:
\begin{itemize}
  \item Relationship between dynamical properties and computational power
  \item Required connections to input layer, how much perturbation required?
\end{itemize}

Advantages include being MUCH less complex than Echo-state networks and similar, as those require operations such as 'multiplication' which is orders more complex than the simple lookup-table transitions for the RBN nodes.


\begin{itemize}
  \item LSM / ESN
  \item 2013-RBN Paper
  \item RBN-Kauffmann networks
  \item Sipper programming complexity (Fra TDT1/22) Snakk om vanskeligheten av å programmere
\end{itemize}
