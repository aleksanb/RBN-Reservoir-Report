\section{Experiments}

\subsection{Creating functioning RBN reservoir systems}

In \cite{rbn-reservoir} the authors find a relationship between computational capability and performance,
that certain values of $L$ give the best reservoir performance,
and that RBNs with connectivity $\langle K \rangle =2$ should outperform other connectivities.
To test these assertions,
we must create and benchmark a number of functioning RBN reservoir systems,
noting their accuracy on the chosen task as well as their computational capability.
Plotting these values against each other will shed light on whether their claims strike true.

We will be using two versions of the \textit{temporal parity} task
(as specified in table \ref{table:task-parameters}) to measure reservoir performance.
The temporal parity task is chosen over temporal density as it is the more difficult task
(shown in \cite{rbn-reservoir}), presumably resulting in more interesting and rich resevoirs.

Computational Capability will be measured with $T=100$ for both versions of the task,
but the required memory $t$ equal to the tasks window size $N$.

\begin{table}
  \centering
  \caption{Task parameters}
  \label{table:task-parameters}
  \begin{tabular}{ll}
    Task type         & Temporal Parity \\
    Num. datasets     & 10              \\
    Dataset length    & 200             \\
    $N$ (window size) & 3 and 5         \\
    $t$ (offset)      & 0               \\
  \end{tabular}
\end{table}


As the number of different RBNs is oppressively large,
$(\frac{2^{2^{K}}N!}{(N-K)!})^N$ \cite{gershenson2004introduction},
we therefore create 30 random specimens for each combination of the RBN parameters displayed in table \ref{table:rbn-combinations}.
This results in 300 samples for each RBN N-K combination for each task.
Note that the created RBNs are of homogenous connectivity,
as opposed to the heterogenous connectivity used \cite{rbn-reservoir}.

\begin{table}
  \centering
  \caption{RBN combinations}
  \label{table:rbn-combinations}
  \begin{tabular}{ll}
    N (nodes)              & 100             \\
    K (connectivity)       & 1, 2, 3         \\
    L (input connectivity) & 0, 10, ..., 100 \\
    Temporal parity        & $N=3,5$ \\
  \end{tabular}
\end{table}

\subsection{Evolving functionally equivalent RBN reservoirs for existing readout layers}
\label{section:experiments:evolving}

We use the genetic algorithm presented in section \label{section:method:evolving-rbns} to evolve functionally equivalent RBNs (hyperparameters as in table \ref{table:ga-hyperparameters}).

We now pick 5 RBN reservoir systems,
chosen for their $> 99.5\%$ accuracy on the Temporal Parity task shown in table
\ref{table:ga-task-parameters},
and store their trained readout layers.
All 5 RBNs have $N=100, K=2$, three of the RBNs having an input connectivity of $L=50$,
the remaining two $L=70$.
They will from now on be referred to as L50\#1, L50\#2, L50\#3, L70\#1, and L70\#2.
This will allow us to measure if the functionally equivalent reservoirs for the readout layers trained on reservoirs with $L=50$ differ from the ones with $L=70$.

\begin{table}
  \centering
  \caption{GA Task parameters}
  \label{table:ga-task-parameters}
  \begin{tabular}{ll}
    Task type         & Temporal Parity \\
    Num. datasets     & 10               \\
    Dataset length    & 200             \\
    $N$ (window size) & 3               \\
    $t$ (offset)      & 0               \\
  \end{tabular}
\end{table}

For each readout layer we create a new instance of the Temporal Parity task used to train it originally.
We then run the GA 30 times with this readout layer–dataset combination,
sampling only the best specimen from each final generation.
This to keep our influence from evolutionary inbreeding to a minimum.
