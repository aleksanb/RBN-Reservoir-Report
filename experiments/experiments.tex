\section{Experiments}

\subsection{Reproduce previous results}

In \cite{rbn-reservoir} the authors find a relationship between computational capability and performance,
that certain values of $L$ give the best reservoir performance,
and that RBNs with connectivity $\langle K \rangle =2$ should outperform other connectivities.
To test these assertions,
we must create and benchmark a number of functioning RBN reservoir systems,
noting their accuracy on the chosen task as well as their computational capability.

We will be using two versions of the \textit{temporal parity} task
(as specified in table \ref{table:task-parameters}) to measure reservoir performance.
The temporal parity task is chosen over temporal density as it is the more difficult task
(shown in \cite{rbn-reservoir}), presumably resulting in more interesting and rich resevoirs.

\begin{table}
  \centering
  \caption{Task parameters}
  \label{table:task-parameters}
  \begin{tabular}{ll}
    Task type         & Temporal Parity \\
    Num. datasets     & 10              \\
    Dataset length    & 200             \\
    $N$ (window size) & 3 and 5         \\
    $t$ (offset)      & 0               \\
  \end{tabular}
\end{table}


As the number of different RBNs is oppressively large,
$(\frac{2^{2^{K}}N!}{(N-K)!})^N$ \cite{gershenson2004introduction},
we therefore create 30 random specimens for each combination of the RBN parameters displayed in table \ref{table:rbn-combinations}.

\begin{table}
  \centering
  \caption{RBN combinations}
  \label{table:rbn-combinations}
  \begin{tabular}{ll}
    N (nodes)              & 100             \\
    K (connectivity)       & 1, 2, 3         \\
    L (input connectivity) & 0, 10, ..., 100 \\
    Temporal parity        & $N=3,5$ \\
  \end{tabular}
\end{table}

\subsection{Evolving reservoirs to re-use existing readout layers}
