\section{Method}

\subsection{Creation of RBN-Reservoirs, the perturbing thereof}

RBNs are initialized randomly, and connected to a Ridge regression readout layer \cite{WhatIsRidgeRegression?}.
They are then trained on the tasks from the next subsection.

In literature, one frequently 'perturbs' or 'annoys' RBNs to see how change spreads through the network. \cite{CitationMissing}
RBNs usually don't take input, but we can extend this Perturbation model to make the RBN handle input as follows:

\begin{itemize}
  \item Initialize RBN, set state to all 0's
  \item LoopStart: Set input-connected nodes to input value
  \item Run RBN as CRBN for 1 time-step, logging the RBN state
  \item If input left, goto loopstart, else give all RBN states to next node in flow
\end{itemize}

The Ridge-Regression layer then receives all the intermediate states, as well as what each of the states should be classified as, and attempts to do regression on this dataset.

If the RBN-Reservoir has 'good' dynamics,
the readout layer will be able to correctly classify the input stream.
Such reservoir / readout layers are then stored, or 'pickled' to disk for later analysis and re-use.

\subsection{Tasks}

Here are the datasets used in this report, originally proposed in \cite{rbn-reservoir}.

\begin{itemize}
  \item Temporal Parity
  \item Temporal Density
\end{itemize}

Here there'll be a somewhat compact explanation of the tasks and required reservoir complexity,
as measured in that previous papaer.

\subsection{Evolving RBN-Reservoirs}

\begin{itemize}
  \item Here we present how a Genetic Algorithm works
  \item Here we present the Genome used in the task
  \item Here we present the fitness function used
\end{itemize}

Note that we allow variable-input connections, even though the readout-layer was trained on a RBN with fixed number of input-connections.

\subsection{Measuring dynamics}

\begin{itemize}
  \item
    How do i estimate computational power?
    Use the complexity measure from \cite{rbn-reservoir} of course!
  \item
    How do i estimate transient times and attractor lengths?
    Random samling! Link to some algorithm i use here
  \item
    Fun Anecdote: Link to that paper measuring power usage in RBN-Reservoirs:
    \cite{rbn-reservoir-energy-efficiency}
\end{itemize}

\subsection{Weaknesses}

\begin{itemize}
  \item Extremely large state-space
  \item Can be difficult to generalize results
  \item
    Genome representation, inefficient, better with binary string?
    Luckily dominated by RBN-Reservoir simulation time
\end{itemize}


\subsection{Experimental setup}

Should i show the RBN-Reservoir performance in solving time-series tasks here instead of as the first section of results?
