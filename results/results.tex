\section{Results}

\subsection{Using RBN-reservoirs in solving time-series tasks}

Turns out it works great, as in that 2013 paper.
There seems to be extremely many usable reservoirs, at least for K=2.

\begin{itemize}
  \item Solving the Temporal-Parity task
  \item Solving the Temporal-Density task
\end{itemize}

\subsection{Evolving new RBN-Reservoirs to use existing readout layers}

For some readout layers (4 or so), check if there are multiple acceptable reservoirs for each readout layer.

\begin{itemize}
  \item Show Fitness Graphs with increasing fitness for evolved RBNs
  \item Complexity analysis of RBNs utilizing the same readout layer
\end{itemize}

We see that there are many acceptable RBN-Reservoirs for each readout-layer.
Maybe these RBN-Reservoirs have similar dynamics?


\subsection{Analysis of Dynamics of developed RBN-Reservoirs}

For each group of evolved RBN-Reservoirs (5 or so in each):

\begin{itemize}
  \item Show Computational complexity measure as developed in that 2013 paper
  \item Show transient time for the group
  \item Show Attractor lengths for the group
\end{itemize}


Perhaps the required dynamics are different for the different tasks?
Remember=3 gives different attractors than remember=4. That'd be cool.


Here there'll be an image visualizing one of the groups of evolved RBN-Reservoirs.
Perhaps we can visually identify similarities.
