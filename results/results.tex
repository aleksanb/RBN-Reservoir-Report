\section{Results}

\begin{figure}
  \resizebox{\columnwidth}{!}{
    \subfloat[100-3]{
      \myboxplot{
% L: 0
\addplot[
boxplot prepared={
    draw position=0,
    median=0.525,
    upper quartile=0.525,
    lower quartile=0.521,
    upper whisker=0.525,
    lower whisker=0.525
},
] coordinates {};
% L: 10
\addplot[
boxplot prepared={
    draw position=1,
    median=0.5175,
    upper quartile=0.575,
    lower quartile=0.48,
    upper whisker=0.775,
    lower whisker=0.455
},
] coordinates {};
% L: 20
\addplot[
boxplot prepared={
    draw position=2,
    median=0.6625,
    upper quartile=0.92375,
    lower quartile=0.51875,
    upper whisker=1.0,
    lower whisker=0.46
},
] coordinates {};
% L: 30
\addplot[
boxplot prepared={
    draw position=3,
    median=0.8975,
    upper quartile=1.0,
    lower quartile=0.5275,
    upper whisker=1.0,
    lower whisker=0.445
},
] coordinates {};
% L: 40
\addplot[
boxplot prepared={
    draw position=4,
    median=1.0,
    upper quartile=1.0,
    lower quartile=0.84875,
    upper whisker=1.0,
    lower whisker=0.435
},
] coordinates {};
% L: 50
\addplot[
boxplot prepared={
    draw position=5,
    median=1.0,
    upper quartile=1.0,
    lower quartile=0.996,
    upper whisker=1.0,
    lower whisker=0.615
},
] coordinates {};
% L: 60
\addplot[
boxplot prepared={
    draw position=6,
    median=1.0,
    upper quartile=1.0,
    lower quartile=0.996,
    upper whisker=1.0,
    lower whisker=0.33
},
] coordinates {};
% L: 70
\addplot[
boxplot prepared={
    draw position=7,
    median=0.875,
    upper quartile=1.0,
    lower quartile=0.54125,
    upper whisker=1.0,
    lower whisker=0.31
},
] coordinates {};
% L: 80
\addplot[
boxplot prepared={
    draw position=8,
    median=0.7625,
    upper quartile=0.92375,
    lower quartile=0.51125,
    upper whisker=1.0,
    lower whisker=0.355
},
] coordinates {};
% L: 90
\addplot[
boxplot prepared={
    draw position=9,
    median=0.51,
    upper quartile=0.74,
    lower quartile=0.51,
    upper whisker=1.0,
    lower whisker=0.505
},
] coordinates {};
% L: 100
\addplot[
boxplot prepared={
    draw position=10,
    median=0.51,
    upper quartile=0.51,
    lower quartile=0.506,
    upper whisker=0.51,
    lower whisker=0.51
},
] coordinates {};
}{0.1}

    }
  %}
  %\resizebox{0.5\columnwidth}{!}{
    \subfloat[100-5]{
      \myboxplot{
% L: 0
\addplot[
boxplot prepared={
    draw position=0,
    median=0.565,
    upper quartile=0.565,
    lower quartile=0.561,
    upper whisker=0.565,
    lower whisker=0.565
},
] coordinates {};
% L: 10
\addplot[
boxplot prepared={
    draw position=1,
    median=0.55,
    upper quartile=0.56875,
    lower quartile=0.52625,
    upper whisker=0.595,
    lower whisker=0.5
},
] coordinates {};
% L: 20
\addplot[
boxplot prepared={
    draw position=2,
    median=0.535,
    upper quartile=0.57375,
    lower quartile=0.52625,
    upper whisker=0.67,
    lower whisker=0.51
},
] coordinates {};
% L: 30
\addplot[
boxplot prepared={
    draw position=3,
    median=0.5525,
    upper quartile=0.565,
    lower quartile=0.5325,
    upper whisker=0.775,
    lower whisker=0.505
},
] coordinates {};
% L: 40
\addplot[
boxplot prepared={
    draw position=4,
    median=0.5575,
    upper quartile=0.58125,
    lower quartile=0.54125,
    upper whisker=0.985,
    lower whisker=0.51
},
] coordinates {};
% L: 50
\addplot[
boxplot prepared={
    draw position=5,
    median=0.56,
    upper quartile=0.585,
    lower quartile=0.555,
    upper whisker=0.845,
    lower whisker=0.515
},
] coordinates {};
% L: 60
\addplot[
boxplot prepared={
    draw position=6,
    median=0.555,
    upper quartile=0.57375,
    lower quartile=0.555,
    upper whisker=0.75,
    lower whisker=0.52
},
] coordinates {};
% L: 70
\addplot[
boxplot prepared={
    draw position=7,
    median=0.5625,
    upper quartile=0.565,
    lower quartile=0.555,
    upper whisker=0.61,
    lower whisker=0.52
},
] coordinates {};
% L: 80
\addplot[
boxplot prepared={
    draw position=8,
    median=0.565,
    upper quartile=0.565,
    lower quartile=0.555,
    upper whisker=0.605,
    lower whisker=0.545
},
] coordinates {};
% L: 90
\addplot[
boxplot prepared={
    draw position=9,
    median=0.49,
    upper quartile=0.51,
    lower quartile=0.49,
    upper whisker=0.525,
    lower whisker=0.46
},
] coordinates {};
% L: 100
\addplot[
boxplot prepared={
    draw position=10,
    median=0.53,
    upper quartile=0.53,
    lower quartile=0.526,
    upper whisker=0.53,
    lower whisker=0.53
},
] coordinates {};
}{0.1}

    }
  }

  \resizebox{\columnwidth}{!}{
    \subfloat[200-3]{
      \myboxplot{
% L: 0
\addplot[
boxplot prepared={
    draw position=0,
    median=0.5,
    upper quartile=0.5,
    lower quartile=0.496,
    upper whisker=0.5,
    lower whisker=0.5
},
] coordinates {};
% L: 20
\addplot[
boxplot prepared={
    draw position=1,
    median=0.5475,
    upper quartile=0.57,
    lower quartile=0.5175,
    upper whisker=0.99,
    lower whisker=0.49
},
] coordinates {};
% L: 40
\addplot[
boxplot prepared={
    draw position=2,
    median=0.99,
    upper quartile=0.99,
    lower quartile=0.81125,
    upper whisker=0.99,
    lower whisker=0.495
},
] coordinates {};
% L: 60
\addplot[
boxplot prepared={
    draw position=3,
    median=0.99,
    upper quartile=0.99,
    lower quartile=0.986,
    upper whisker=0.99,
    lower whisker=0.665
},
] coordinates {};
% L: 80
\addplot[
boxplot prepared={
    draw position=4,
    median=0.99,
    upper quartile=0.99,
    lower quartile=0.986,
    upper whisker=0.99,
    lower whisker=0.99
},
] coordinates {};
% L: 100
\addplot[
boxplot prepared={
    draw position=5,
    median=0.99,
    upper quartile=0.99,
    lower quartile=0.986,
    upper whisker=0.99,
    lower whisker=0.99
},
] coordinates {};
% L: 120
\addplot[
boxplot prepared={
    draw position=6,
    median=0.99,
    upper quartile=0.99,
    lower quartile=0.986,
    upper whisker=0.99,
    lower whisker=0.99
},
] coordinates {};
% L: 140
\addplot[
boxplot prepared={
    draw position=7,
    median=0.99,
    upper quartile=0.99,
    lower quartile=0.986,
    upper whisker=0.99,
    lower whisker=0.565
},
] coordinates {};
% L: 160
\addplot[
boxplot prepared={
    draw position=8,
    median=0.99,
    upper quartile=0.99,
    lower quartile=0.785,
    upper whisker=0.99,
    lower whisker=0.43
},
] coordinates {};
% L: 180
\addplot[
boxplot prepared={
    draw position=9,
    median=0.7225,
    upper quartile=0.8125,
    lower quartile=0.555,
    upper whisker=0.99,
    lower whisker=0.43
},
] coordinates {};
% L: 200
\addplot[
boxplot prepared={
    draw position=10,
    median=0.5,
    upper quartile=0.5,
    lower quartile=0.496,
    upper whisker=0.5,
    lower whisker=0.5
},
] coordinates {};
}{0.05}

    }
  %}
  %\resizebox{0.5\columnwidth}{!}{
    \subfloat[200-5]{
      \myboxplot{
% L: 0
\addplot[
boxplot prepared={
    draw position=0,
    median=0.55,
    upper quartile=0.55,
    lower quartile=0.546,
    upper whisker=0.55,
    lower whisker=0.55
},
] coordinates {};
% L: 20
\addplot[
boxplot prepared={
    draw position=1,
    median=0.5375,
    upper quartile=0.55,
    lower quartile=0.53,
    upper whisker=0.59,
    lower whisker=0.485
},
] coordinates {};
% L: 40
\addplot[
boxplot prepared={
    draw position=2,
    median=0.5475,
    upper quartile=0.57875,
    lower quartile=0.53,
    upper whisker=0.715,
    lower whisker=0.48
},
] coordinates {};
% L: 60
\addplot[
boxplot prepared={
    draw position=3,
    median=0.5625,
    upper quartile=0.58375,
    lower quartile=0.53875,
    upper whisker=0.81,
    lower whisker=0.505
},
] coordinates {};
% L: 80
\addplot[
boxplot prepared={
    draw position=4,
    median=0.57,
    upper quartile=0.63875,
    lower quartile=0.54,
    upper whisker=0.855,
    lower whisker=0.51
},
] coordinates {};
% L: 100
\addplot[
boxplot prepared={
    draw position=5,
    median=0.5725,
    upper quartile=0.6275,
    lower quartile=0.55125,
    upper whisker=0.805,
    lower whisker=0.44
},
] coordinates {};
% L: 120
\addplot[
boxplot prepared={
    draw position=6,
    median=0.58,
    upper quartile=0.62875,
    lower quartile=0.56,
    upper whisker=0.795,
    lower whisker=0.525
},
] coordinates {};
% L: 140
\addplot[
boxplot prepared={
    draw position=7,
    median=0.56,
    upper quartile=0.57375,
    lower quartile=0.54625,
    upper whisker=0.99,
    lower whisker=0.44
},
] coordinates {};
% L: 160
\addplot[
boxplot prepared={
    draw position=8,
    median=0.56,
    upper quartile=0.56,
    lower quartile=0.55,
    upper whisker=0.615,
    lower whisker=0.485
},
] coordinates {};
% L: 180
\addplot[
boxplot prepared={
    draw position=9,
    median=0.56,
    upper quartile=0.56,
    lower quartile=0.556,
    upper whisker=0.6,
    lower whisker=0.51
},
] coordinates {};
% L: 200
\addplot[
boxplot prepared={
    draw position=10,
    median=0.55,
    upper quartile=0.55,
    lower quartile=0.546,
    upper whisker=0.55,
    lower whisker=0.55
},
] coordinates {};
}{0.05}

    }
  }
\end{figure}

\begin{figure}
  \resizebox{\columnwidth}{!}{
    \subfloat[200-3]{
      \myboxplot{
% L: 0
\addplot[
boxplot prepared={
    draw position=0,
    median=0.5,
    upper quartile=0.5,
    lower quartile=0.496,
    upper whisker=0.5,
    lower whisker=0.5
},
] coordinates {};
% L: 20
\addplot[
boxplot prepared={
    draw position=1,
    median=0.5475,
    upper quartile=0.57,
    lower quartile=0.5175,
    upper whisker=0.99,
    lower whisker=0.49
},
] coordinates {};
% L: 40
\addplot[
boxplot prepared={
    draw position=2,
    median=0.99,
    upper quartile=0.99,
    lower quartile=0.81125,
    upper whisker=0.99,
    lower whisker=0.495
},
] coordinates {};
% L: 60
\addplot[
boxplot prepared={
    draw position=3,
    median=0.99,
    upper quartile=0.99,
    lower quartile=0.986,
    upper whisker=0.99,
    lower whisker=0.665
},
] coordinates {};
% L: 80
\addplot[
boxplot prepared={
    draw position=4,
    median=0.99,
    upper quartile=0.99,
    lower quartile=0.986,
    upper whisker=0.99,
    lower whisker=0.99
},
] coordinates {};
% L: 100
\addplot[
boxplot prepared={
    draw position=5,
    median=0.99,
    upper quartile=0.99,
    lower quartile=0.986,
    upper whisker=0.99,
    lower whisker=0.99
},
] coordinates {};
% L: 120
\addplot[
boxplot prepared={
    draw position=6,
    median=0.99,
    upper quartile=0.99,
    lower quartile=0.986,
    upper whisker=0.99,
    lower whisker=0.99
},
] coordinates {};
% L: 140
\addplot[
boxplot prepared={
    draw position=7,
    median=0.99,
    upper quartile=0.99,
    lower quartile=0.986,
    upper whisker=0.99,
    lower whisker=0.565
},
] coordinates {};
% L: 160
\addplot[
boxplot prepared={
    draw position=8,
    median=0.99,
    upper quartile=0.99,
    lower quartile=0.785,
    upper whisker=0.99,
    lower whisker=0.43
},
] coordinates {};
% L: 180
\addplot[
boxplot prepared={
    draw position=9,
    median=0.7225,
    upper quartile=0.8125,
    lower quartile=0.555,
    upper whisker=0.99,
    lower whisker=0.43
},
] coordinates {};
% L: 200
\addplot[
boxplot prepared={
    draw position=10,
    median=0.5,
    upper quartile=0.5,
    lower quartile=0.496,
    upper whisker=0.5,
    lower whisker=0.5
},
] coordinates {};
}{0.05}

    }
    \subfloat[200-5]{
      \myboxplot{
% L: 0
\addplot[
boxplot prepared={
    draw position=0,
    median=0.55,
    upper quartile=0.55,
    lower quartile=0.546,
    upper whisker=0.55,
    lower whisker=0.55
},
] coordinates {};
% L: 20
\addplot[
boxplot prepared={
    draw position=1,
    median=0.5375,
    upper quartile=0.55,
    lower quartile=0.53,
    upper whisker=0.59,
    lower whisker=0.485
},
] coordinates {};
% L: 40
\addplot[
boxplot prepared={
    draw position=2,
    median=0.5475,
    upper quartile=0.57875,
    lower quartile=0.53,
    upper whisker=0.715,
    lower whisker=0.48
},
] coordinates {};
% L: 60
\addplot[
boxplot prepared={
    draw position=3,
    median=0.5625,
    upper quartile=0.58375,
    lower quartile=0.53875,
    upper whisker=0.81,
    lower whisker=0.505
},
] coordinates {};
% L: 80
\addplot[
boxplot prepared={
    draw position=4,
    median=0.57,
    upper quartile=0.63875,
    lower quartile=0.54,
    upper whisker=0.855,
    lower whisker=0.51
},
] coordinates {};
% L: 100
\addplot[
boxplot prepared={
    draw position=5,
    median=0.5725,
    upper quartile=0.6275,
    lower quartile=0.55125,
    upper whisker=0.805,
    lower whisker=0.44
},
] coordinates {};
% L: 120
\addplot[
boxplot prepared={
    draw position=6,
    median=0.58,
    upper quartile=0.62875,
    lower quartile=0.56,
    upper whisker=0.795,
    lower whisker=0.525
},
] coordinates {};
% L: 140
\addplot[
boxplot prepared={
    draw position=7,
    median=0.56,
    upper quartile=0.57375,
    lower quartile=0.54625,
    upper whisker=0.99,
    lower whisker=0.44
},
] coordinates {};
% L: 160
\addplot[
boxplot prepared={
    draw position=8,
    median=0.56,
    upper quartile=0.56,
    lower quartile=0.55,
    upper whisker=0.615,
    lower whisker=0.485
},
] coordinates {};
% L: 180
\addplot[
boxplot prepared={
    draw position=9,
    median=0.56,
    upper quartile=0.56,
    lower quartile=0.556,
    upper whisker=0.6,
    lower whisker=0.51
},
] coordinates {};
% L: 200
\addplot[
boxplot prepared={
    draw position=10,
    median=0.55,
    upper quartile=0.55,
    lower quartile=0.546,
    upper whisker=0.55,
    lower whisker=0.55
},
] coordinates {};
}{0.05}

    }
  }
\end{figure}

\subsection{Using RBN-reservoirs in solving time-series tasks}

Turns out it works great, as in that 2013 paper.
There seems to be extremely many usable reservoirs, at least for K=2.

\begin{itemize}
  \item Solving the Temporal-Parity task
  \item Solving the Temporal-Density task
\end{itemize}

\subsection{Evolving new RBN-Reservoirs to use existing readout layers}

For some readout layers (4 or so), check if there are multiple acceptable reservoirs for each readout layer.

\begin{itemize}
  \item Show Fitness Graphs with increasing fitness for evolved RBNs
  \item Complexity analysis of RBNs utilizing the same readout layer
\end{itemize}

We see that there are many acceptable RBN-Reservoirs for each readout-layer.
Maybe these RBN-Reservoirs have similar dynamics?


\subsection{Analysis of Dynamics of developed RBN-Reservoirs}

For each group of evolved RBN-Reservoirs (5 or so in each):

\begin{itemize}
  \item Show Computational complexity measure as developed in that 2013 paper
  \item Show transient time for the group
  \item Show Attractor lengths for the group
\end{itemize}


Perhaps the required dynamics are different for the different tasks?
Remember=3 gives different attractors than remember=4. That'd be cool.


Here there'll be an image visualizing one of the groups of evolved RBN-Reservoirs.
Perhaps we can visually identify similarities.
