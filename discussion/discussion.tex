\section{Discussion}

\subsection{Creating functioning RBN reservoir systems}

\subsubsection{Temporal Parity with $N=3$}
There is an abundance of well-performing reservoirs for $K=2$ and $K=3$.
Reservoirs with higher connectivity have their fitnesses peak at higher values of input connectivity.
This is in line with the expectations from \cite{rbn-reservoir} described in section \ref{section:optimal-perturbance}.
The more chaotic the dynamics of the reservoir,
the more it has to be perturbed to retain the input information.

\subsubsection{Temporal Parity with $N=5$}
The distribution of well-performing reservoirs is shifted heavily towards $K=3$.
There are single outliers that perform well for $K=2$,
but the accuracy medians are considerably lower than for the reservoirs with $K=3$.

Comparing the observed performance against the performace of the RRC systems benchmarked in \cite{rbn-reservoir} on the same task,
one sees the same drop in performance from Temporal Parity with $N=3$ to $N=5$.

\subsubsection{Computational Capability}
Comparing the Computational Capability plots for both tasks (Figures \ref{figure:results:temporal-parity-3} and \ref{figure:results:temporal-parity-5}),
one observes that the individuals with a higher Computational Capability have a tendency to be shifted towards higher accuracies.
The lower accuracy bounds are increased as compared to the samples located at $.0CC$.
This indicates a positive correlation between Computational Capability and performance,
again in line with the expectations from \cite{rbn-reservoir}.
One might be quick to assume that reservoirs achieving no higher than a $0.5$ accuracy perform no better than the tossing of a coin.
There is however no guarantee that the distribution of correct classifications follow a binomial model with $p=0.5$.

\subsubsection{Optimal Connectivity}
When comparing general performance across $K$ values on the two tasks,
the optimal connectivitity seems to be closer to $K=3$ than $K=2$.
As described in section \ref{section:rbns},
critical dynamics for heterogenous networks should be the most frequent at an average $\langle K \rangle = 2$.
Such networks can therefore contain subgraphs of both higher and lower connectivity,
while keeping the average connectivity the same.
This disparity can therefore be explained by the fact that an homogenous RBN with $K=3$ can emulate an RBN with lower connectivity by having two or more in-edges from the same ancestor node,
while an homogenous RBN with $K=2$ cannot emulate a higher-connectivity one.

\subsection{Evolving functionally equivalent RBN reservoirs for existing readout layers}

There are a great number of functionally equivalent reservoirs for each functioning readout layer,
and the Genetic Algorithm finds them efficiently.
Sadly, a tight correlation between the properties of the RBN from the original RRC system,
and RBNs evolved against its readout layer seem spurious at best.
In fact, the connectivity distributions of Figure \ref{figure:evolved-connectivity} and computational capabilities of Figure \ref{figure:evolved-ccs} seem much more representative of the general RBN population, as shown in Figure \ref{fig:res:d-100-3-2}.
This indicates that while there are many compatible reservoirs for a given readout layer,
the distribution of the reservoirs are likely the same as the distribution of reservoirs with the same connectivity in general.

It should be noted that during earlier simulations with a per-component mutation rate of $1\%$ as opposed to the $10\%$ actually used,
the mean of the fitness distribution during GA runs would be much closer to the best specimens,
but would frequently get stuck in a local maxima with fitness around 0.77, eventually timing out.
This suggests that a lack of genetic variety was the culprit,
as it's limited what genome crossover can accomplish alone.
Increasing the rate to $10\%$ drastically increased convergence rates (as shown in Figure \ref{figure:termination-generations},
even though the population median stays close to 0.5 for the entire run.

Finally, this implies a one-to-many mapping between reservoirs and readout layers,
as there are multiple compatible reservoirs for each previously trained readout layer.
This makes the potential use of a smaller generative genome for evolving RRC systems interesting.
Even though it hits fewer points in the RBN fitness landscape than the fixed genome used in this paper,
a large amount of these points are still usable for each instance of a working readout layer.
