\documentclass[conference]{IEEEtran}

\usepackage[utf8]{inputenc}
\usepackage[T1]{fontenc} % Use 8-bit encoding that has 256 glyphs
\usepackage[english]{babel} % English language/hyphenation
\usepackage[caption=false,font=footnotesize]{subfig}

\ifCLASSINFOpdf
  \usepackage[pdftex]{graphicx}
  % declare the path(s) where your graphic files are
  \graphicspath{{../pdf/}{../jpeg/}{../static/}}
  % and their extensions so you won't have to specify these with
  % every instance of \includegraphics
  \DeclareGraphicsExtensions{.pdf,.jpeg,.png}
\else
\fi

\usepackage{array}

\usepackage{fixltx2e}

\usepackage{url}

\usepackage{amsmath}

\hyphenation{op-tical net-works semi-conduc-tor}

\usepackage[backend=bibtex]{biblatex}
\bibliography{IEEEabrv,bibfiles/bibtexlibs}

\usepackage{todonotes}
\usepackage{bytefield}

\newcommand{\cm}[1]{
  \todo[inline, color=cyan]{Citation missing: #1}
}

\usepackage{tikz}
\usetikzlibrary{arrows}
\tikzset{vertex/.style = {shape=circle,draw,minimum size=2em}}
\tikzset{edge/.style = {->, > = latex'}}
\tikzset{node distance = 5em}


\begin{document}

% paper title
% Titles are generally capitalized except for words such as a, an, and, as,
% at, but, by, for, in, nor, of, on, or, the, to and up, which are usually
% not capitalized unless they are the first or last word of the title.
% Linebreaks \\ can be used within to get better formatting as desired.
% Do not put math or special symbols in the title.
\title{Reservoir computing med greier?}

\author{
    \IEEEauthorblockN{Aleksander Vognild Burkow}
    \IEEEauthorblockA{
        Department of Computer and Information Science\\
        Norwegian University of Science and Technology\\
        Sem Sælandsvei 7-9, 7491 Trondheim, Norway\\
        \texttt{aleksanderburkow@gmail.com}
    }
}

\maketitle

\begin{abstract}

Reservoir Computing, a relatively new approach to machine learning,
utilizes untrained Recurrent Neural Nets as a reservoir of dynamics to preprocess some temporal task,
making it separable with a linear readout layer
Originating from the study of Liquid State Machines and Echo State Networks,
potentially any sparsely connected network containing feedforward and feedback loops can be a reservoir.
Random Boolean Networks (RBN) is such a sparsely connected network that may be suitable for Reservoir Computing.

In this paper we investigate the dynamics, performance, and viability of RBNs used for Reservoir Computing (RRC).
A system to investigate these properties is implemented,
and its correctness is validated by comparing its results with those of comparable studies.
The chosen reproduced experiments result in the following findings:
The more chaotic the phase of an RBN is, the higher its required input connectivity.
The value of $K$ which provides optimal computational power is found to lie closer to $K=3$ when using homogenous networks,
as opposed to the heterogenous optimal $\langle K \rangle = 2$.
A relationship between Computational Capability and actual reservoir performance seems to exist.

Finally, we find a one-to-many mapping between the readout layer in an already-trained RRC system and different RBN reservoirs,
with there being a seemingly large set of interchangeable reservoirs for each readout layer.
This makes the potential use of a smaller generative genome for evolving RRC systems interesting.
Even though it hits fewer points in the RBN fitness landscape than the fixed genome used in this paper,
a large amount of these points are still usable for each instance of a working readout layer.

\end{abstract}


\section{Introduction}

Reservoir computing (RC) is a form of machine learning that sprung out from the study of recurrent neural networks (RNNs).
In short, it utilizes the dynamics of some complex system dubbed a 'reservoir' to preprocess a timeseries problem,
transforming it from a temporal to a spacial one in the reservoir, making it then separable with a usually simple readout layer.

In this paper we investigate the dynamics of Reservoir Computing systems where the reservoir is a Random Boolean Network (RBN) \cite{gershenson2004introduction},
an approach found fruitful in \cite{rbn-reservoir}.

First we reproduce chosen experiments from \cite{rbn-reservoir}, creating a working RBN-reservoir computing system (RRC).

Next we investigate whether the readout layer of a working RRC system can be re-used with other RBN-reservoirs than the one it was trained on,
and still stay accurate on the original classification task.
These functionally equivalent reservoirs will be found through the use of an evolutionary algorithm.

Last we look at the dynamics and characteristics of these groups of RBN-Reservoirs, attempting to find any similarities that might be exploitable.

\subsection{A brief introduction to reservoir computing}

Recurrent neural networks, as opposed to feed-forward neural networks,
are notoriously time consuming and difficult to train.
This due to feedback from the recurrent connections during the training process,
allowing small topology changes to drastically change ones position in the fitness landscape.

It was therefore proposed both in \cite{jaeger2002adaptive} (as echo state networks, or ESN)
and \cite{natschlager2002liquid} (as liquid state machines, or LSM) to separate the RNN into two parts,
the untrained reccurrent reservoir, and the trained readout layer.
Both of these methods have been unified into the field of Reservoir Computing,
now focusing on the separate training and evolution of the recurrent and readout part \cite{lukovsevivcius2012reservoir}.

Useful RC resources include Organic \cite{organic}, an online RC hub,
providing documentation, references, and a RC toolbox implemented in python.
It has been quite useful for this project.
Exiting applications of reservoir computing include speech and handwriting recognition,
as well as controlling robotics, as detailed in \cite{lukovsevivcius2012reservoir}.

\subsection{Alternatives to classical reservoirs}

An interesting question arises:
Are there other types of complex systems that can be used as reservoirs?
What properties must these reservoirs have to be able to solve problems?

Complex networks similar to the sparsely connected RNNs found in classical RC devices include Cellular Automata and Random Boolean Networks.
Both are considerably less complex than RNNs, requiring only simple lookup tables for state transitions as opposed to multiplication for RNNs.
A short treatise on Cellular Computing \& friends is available here \cite{sipper1999emergence}.
Cellular computing provides a potentially powerful alternative to classical computers,
leveraging extreme parallelism, simple components and local state.

The ever-so-cited 'Water bucket' paper investigated the use of an actual bucket of water as a reservoir \cite{fernando2003pattern},
successfully recognizing patterns and achieving decent performance at that.
The RBN-reservoir \cite{rbn-reservoir} approach mentioned earlier, has also been found to be viable.

\subsection{Dynamics and complexity of reservoirs}

In general, one wishes to use a reservoir with a complexity matching that of the problem.
Utilizing a modern processor to predict a very simple timeseries would be a classic case of shooting sparrows with cannons \cite{wiki:sparrow},
while achieving the same using a RBN or CA might be more impressive.

\todo[inline]{Gotta finish introduction from hereon and down}

So we have all this litterature on complex systems (CITATIONS HERE) that show that optimal computing power is most often found on the 'edge of chaos', or the critical state, on the edge between periodic and chaotic dynamics.
In addition we have a measure for computational power in RRN's developed in \cite{rbn-reservoir} that awards good scores to networks that are able to both separate two at-some-point similar input sequences, as well as two sequences that used to differ in the past.
They show that this measure too, is maximized at critical connectivity (K=2) for RBNs.

% HERE I WRITTEN TTTO

These reservoirs have dynamics which can be analysed.
So if we find other networks with different topologies but similar dynamics,
we should be able to use this new network as the 'reservoir' with a simple readout layer.

If this area of research remains fruitful,
it can be used to develop metrics for what other 'reservoirs' or even physical materials might be useful as a computational base.
This can be useful for evolution in materio, for selecting substrates \cite{evolutionInMaterio}

\subsection{What i actually done in this paper?}

A rbn-simulator was developed in the Python programming language,
using the following things for glue, regression, inspiration, and duct-tape.

\begin{itemize}
  \item MDP + Oger for putting nodes in a flow, and regressing
  \item Numpy + Scipy for numerical calculations, array backing
  \item RBN Matlab toolbox for inspiration
  \item Homegrown RBN-simulator, Evolutionary alogrithm borrowed from friend
\end{itemize}


% An example of a floating figure using the graphicx package.
% Note that \label must occur AFTER (or within) \caption.
% For figures, \caption should occur after the \includegraphics.
% Note that IEEEtran v1.7 and later has special internal code that
% is designed to preserve the operation of \label within \caption
% even when the captionsoff option is in effect. However, because
% of issues like this, it may be the safest practice to put all your
% \label just after \caption rather than within \caption{}.
%
% Reminder: the "draftcls" or "draftclsnofoot", not "draft", class
% option should be used if it is desired that the figures are to be
% displayed while in draft mode.
%
%\begin{figure}[!t]
%\centering
%\includegraphics[width=2.5in]{myfigure}
% where an .eps filename suffix will be assumed under latex,
% and a .pdf suffix will be assumed for pdflatex; or what has been declared
% via \DeclareGraphicsExtensions.
%\caption{Simulation results for the network.}
%\label{fig_sim}
%\end{figure}

% Note that IEEE typically puts floats only at the top, even when this
% results in a large percentage of a column being occupied by floats.


% An example of a double column floating figure using two subfigures.
% (The subfig.sty package must be loaded for this to work.)
% The subfigure \label commands are set within each subfloat command,
% and the \label for the overall figure must come after \caption.
% \hfil is used as a separator to get equal spacing.
% Watch out that the combined width of all the subfigures on a 
% line do not exceed the text width or a line break will occur.
%
%\begin{figure*}[!t]
%\centering
%\subfloat[Case I]{\includegraphics[width=2.5in]{box}%
%\label{fig_first_case}}
%\hfil
%\subfloat[Case II]{\includegraphics[width=2.5in]{box}%
%\label{fig_second_case}}
%\caption{Simulation results for the network.}
%\label{fig_sim}
%\end{figure*}
%
% Note that often IEEE papers with subfigures do not employ subfigure
% captions (using the optional argument to \subfloat[]), but instead will
% reference/describe all of them (a), (b), etc., within the main caption.
% Be aware that for subfig.sty to generate the (a), (b), etc., subfigure
% labels, the optional argument to \subfloat must be present. If a
% subcaption is not desired, just leave its contents blank,
% e.g., \subfloat[].


% An example of a floating table. Note that, for IEEE style tables, the
% \caption command should come BEFORE the table and, given that table
% captions serve much like titles, are usually capitalized except for words
% such as a, an, and, as, at, but, by, for, in, nor, of, on, or, the, to
% and up, which are usually not capitalized unless they are the first or
% last word of the caption. Table text will default to \footnotesize as
% IEEE normally uses this smaller font for tables.
% The \label must come after \caption as always.
%
%\begin{table}[!t]
%% increase table row spacing, adjust to taste
%\renewcommand{\arraystretch}{1.3}
% if using array.sty, it might be a good idea to tweak the value of
% \extrarowheight as needed to properly center the text within the cells
%\caption{An Example of a Table}
%\label{table_example}
%\centering
%% Some packages, such as MDW tools, offer better commands for making tables
%% than the plain LaTeX2e tabular which is used here.
%\begin{tabular}{|c||c|}
%\hline
%One & Two\\
%\hline
%Three & Four\\
%\hline
%\end{tabular}
%\end{table}


% Note that the IEEE does not put floats in the very first column
% - or typically anywhere on the first page for that matter. Also,
% in-text middle ("here") positioning is typically not used, but it
% is allowed and encouraged for Computer Society conferences (but
% not Computer Society journals). Most IEEE journals/conferences use
% top floats exclusively. 
% Note that, LaTeX2e, unlike IEEE journals/conferences, places
% footnotes above bottom floats. This can be corrected via the
% \fnbelowfloat command of the stfloats package.

\section{Future work}

Drodle om hva jeg vil gjøre til våren.

\section{Conclusion}

It has also been suggested 29 that LSMs might present a useful framework
for modeling computations in gene regulation networks. These networks
also compute on time varying inputs (e.g., external signals) and
produce a multitude of time varying output signals (transcription rates of
genes). Furthermore these networks are composed of a very large number
of diverse subprocesses (transcription of transcription factors) that tend to
have each a somewhat different temporal dynamics 30

Kanskje det ja?


\printbibliography

\end{document}
